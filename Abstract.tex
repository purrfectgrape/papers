\documentclass[a4paper]{article}
%\usepackage{simplemargins}

%\usepackage[square]{natbib}
\usepackage{amsmath}
\usepackage{amsfonts}
\usepackage{amssymb}
\usepackage{graphicx}

\begin{document}
\pagenumbering{gobble}

\Large
 \begin{center}
Vietnamese Speakers’ Cues to the Perception of Stress\\ 
\hspace{10pt}

% Author names and affiliations
\large
Giang Le \\

\hspace{10pt}

\small  
University of Illinois at Urbana Champaign \\
gianghl2@illinois.edu \\

\end{center}

\hspace{10pt}

\normalsize

L1 transfer affects the process of L2 acquisition in a significant way, both in perception and production, as learners have a tendency to apply phonological patterns of their native language to the target language.
This study investigates the extent to which Vietnamese native speakers rely on F0 as a primary cue to perceive stress in English nonce words by manipulating the pitch contour around the stressed syllable by creating different environments where such pitch contours are realized, and subsequently measuring the differences in performance of stress location matching as a result. While the acoustic correlates of stress in English are F0, duration, intensity, and vowel quality (Fry, 1955; Libermann, 1960) \cite{Fry:1955aa}, the acoustic correlates of tone in Vietnamese are F0, duration, and voice quality (Pham, 2000; Nguyen \& Edmonton, 1997). Despite some overlapping of acoustic correlates, English lexical stress prediction cannot be predicated on pitch alone. For example, in a rising tonal contour context such as that of a yes/no question (L*H-H\%), English stressed syllable actually receives a low pitch accent (Pierrehumbert, 1980).

The independent variables of this study are stress location in a nonce word, the number of syllables in the stimuli, and the type of intonation context where a statement context corresponds to a falling intonation pattern and a yes/no question context corresponds to a rising intonation pattern. The dependent variable of this study is the number or percentage of correct responses the participants give to a perceptual matching task. To avoid lexical retrieval and memorization effect, the nonce words were selected based on a search in a pronunciation corpus of American English. The nonce words have the same syllable shape as a real English word, follow English phonotactics, and are controlled for factors such as tendency for vowel reduction. Besides the nonce word items, a set of filler items was included in the test instrument, ranging from tokens that are minimal segmental contrast pairs to tokens that differ by syllable length. The experiment was repeated for a control group of L1 American English speakers. The tokens set was recorded by a native American English speaker, randomized during the actual experiment in blocks, and distributed to the participants in different test lists following the Latin square design. The participants listened to the stimuli with varying stress locations three times and then listened to the stimuli with either a statement or yes/no question intonation. They were asked to identify the sound that they heard previously which matches the sound they have just heard. Similar to (Ou, 2010)'s findings, the prediction for this study was that L1 Vietnamese L2 English speakers would show a significant difference in perceiving stress compared to the control group when the word has a yes/no intonation contour, because of Vietnamese speakers' tendency to rely on F0 as an acoustic cue for tone perception.  \\
\bibliographystyle{plain}
\bibliography{CitesforStressPaper}{}
\end{document}